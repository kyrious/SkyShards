\documentclass[]{article}
\usepackage[ngerman]{babel}
\usepackage[utf8]{inputenc}

%opening
\title{Minecraft Mod\\
	Working Title: Sky Shards}
\author{}

\begin{document}

\maketitle

\begin{abstract}

\end{abstract}

\section{Shards-Mechanic}
\begin{itemize}
	\item Ley-Lines fließen durch Welt, Energieströme
	\item Schnittpunkte von Ley-Lines besonders
	\item Jeder Shard braucht Graviton
	\item Graviton braucht Energie, um Blocks zu halten
	\begin{itemize}
		\item Je mehr Energie, desto größerer Radius
		\item Sicherer Radius und halbstabiler Radius
		\item $\frac{r_{halb}}{r_{sicher}} < 0,9$ wird größer mit mehr Energie
		\item Danach nur Maxradius-Steigerung, jedoch wesentlich mehr Energie dafür
		\item Gravitons ziehen Energie aus Ley-Lines, verbessert durch Upgrades
		\item ziehen immer maximal Energie, man kann jedoch Energie abzapfen
	\end{itemize}
	\item Konstrukt zum Spawnen neuer Gravitons
	\begin{itemize}
		\item Param: generelle Richtung, genutzte Resourcen, Tier der Gravitons \\(nicht fest)
		\item entstehen an Schnittpunkten von Ley-Lines
	\end{itemize}
	\item verschiedene Graviton-Typen
	\item verschiedene Biome, je nach Graviton-Typ
	\item je nach Biome/Graviton-Typ neue Resourcen
	\item beeinflussbar durch Konstrukte:
	\begin{itemize}
		\item Shard-Shape (Brücken, etc.)
		\item Biome
	\end{itemize}
	\item Shards können zufällig spawnen
	\begin{itemize}
		\item halten länger als künstlich erzeugte
		\item höhere Chance auf seltene Gravitons/Shards
	\end{itemize}
\end{itemize}

\section{Resourcen und Abbau}
\begin{itemize}
	\item Resourcen Shard-abhängig
	\item Resourcen haben geringere Quantität als in Vanilla, langsamerer Abbau (Splitter?) $\longrightarrow$ mehr pro Block
	\item Konstrukte zum Abbau von Resourcen
	\begin{itemize}
		\item Nicht vollautomatisiert(!), nur halb-automatisiert (Interaktion alle viertel(?) Stunde)
	\end{itemize}
\end{itemize}

\section{Tools}
\begin{itemize}
	\item Neue Durability-Balance (alte Tools entfernen oder anpassen)
	\item Grappling Hook
	\begin{itemize}
		\item Kann Brücken spannen, begrenzte Länge
		\item lange Brücken durch kleine Shards verbunden durch kleine Brücken
	\end{itemize}
	\item Hedron-Battery
	\begin{itemize}
		\item speichern Energie, aufladbar
		\item verschiedene Tiers
	\end{itemize}
\end{itemize}

\section{Random Ideas}
\begin{itemize}
	\item Gravitons haben Pre-/Suffix, craftbar
	\item neue (fliegende) Monster
\end{itemize}
\end{document}
